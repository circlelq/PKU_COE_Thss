\documentclass[UTF8,oneside,a4,12pt]{ctexbook}
% 各种设置都在下面的文件里
\input{setting.tex}
\graphicspath{{fig/}} % 图片的文件路径

% pdf 文件设置
\hypersetup{
   pdfauthor={作者},
   pdftitle={pdf名称}
}

% 在这输入你的标题,会加入页眉
\newcommand\TheTitle{\sf 标题}

% Mac 字体设置
% \setmainfont{TimesNewRomanPSMT}
% \setsansfont{Helvetica-Light}
% \setCJKmainfont[ItalicFont=STKaitiSC-Regular,BoldFont=STSongti-SC-Black]{STSongti-SC-Regular}
% \setCJKsansfont[BoldFont=STHeitiSC-Medium]{STHeitiSC-Light}
% \setCJKmonofont{STKaitiSC-Bold}% 加粗楷体
% \newfontfamily\ktb{STKaitiSC-Bold}

% Win 字体设置
% \setCJKsansfont{SimHei}
% 教务要求黑体必须是SimHei,还请Mac用户下载SimHei字体

% 封面请用 word 转 PDF 后插入

\begin{document}

% 设置图片、表格开头的中文显示
\input{setc.tex}

\setlength{\baselineskip}{20pt} % 行间距

\input{chap/copyright}

\input{chap/abstract}

{
   \hypersetup{linkcolor=black}
   \vspace{100pt}
   \newpage
   \pagenumbering{gobble}
   {\centering{\tableofcontents\thispagestyle{empty}}}
   \clearpage
   \pagenumbering{arabic}
   \setcounter{page}{1}
}

% 页眉页脚设置
\fancypagestyle{fancynohead}{
    \fancyhead{}
    \renewcommand\headrulewidth{0pt}
    \fancyfoot[L]{}
    \fancyfoot[C]{\sf \wuhao 第\thepage 页}
    \fancyfoot[R]{}
}

\pagestyle{fancynohead}
\ctexset{chapter = {pagestyle = fancynohead}}

\input{chap/introduction}

\input{chap/chap1}


% 参考文献
\input{bib.tex}

% 附录
\input{chap/appendix}


% 致谢
\input{chap/thanks}


% 声明
\input{chap/declaration}


\nocite{*}

\end{document}


